\documentclass[11pt,spanish,listoffigures]{tfgetsinf}
\usepackage[utf8]{inputenc} 

%%%%%%%%%%%%%%%%%%%%%%%%%%%%%% PORTADA %%%%%%%%%%%%%%%%%%%%%%%%%%%%%%

\title{Implementación de un proxy inverso para una arquitectura de microservicios}
\author{Alejandro Carrión Sanmartín}
\tutor{Patricio Letelier Torres}
\curs{2020-2021}

%%%%%%%%%%%%%%%%%%%%%%%%%%%%%% PALABRAS CLAVE %%%%%%%%%%%%%%%%%%%%%%%%%%%%%%

\keywords{????, ?????????} % Paraules clau 
         {?????, ???} % Palabras clave
         {?????, ?????} % Key words

\begin{document}

%%%%%%%%%%%%%%%%%%%%%%%%%%%%%% RESUMEN %%%%%%%%%%%%%%%%%%%%%%%%%%%%%%

\begin{abstract}[spanish]
????
\end{abstract}
\begin{abstract}[catalan]
????
\end{abstract}
\begin{abstract}[english]
????
\end{abstract}

\mainmatter

%%%%%%%%%%%%%%%%%%%%%%%%%%%%%% INTRODUCCIÓN %%%%%%%%%%%%%%%%%%%%%%%%%%%%%%

\chapter{Introducción}

El uso de arquitecturas de microservicios es una práctica ya consolidada hoy en día en el mundo del desarrollo de \emph{software}. Esta consiste en la construcción de servicios independientes, ejecutados en procesos diferentes, que se encargan de realizar funciones concretas y que trabajan de forma conjunta para lograr el objetivo u objetivos globales de la aplicación. Los beneficios que otorga este enfoque frente a la aproximación tradicional monolítica son muchos y muy variados. Algunos de ellos son:

\begin{itemize}

	\item \textbf{Uso de diferentes tecnologías}.
Cada microservicio puede estar construido con una tecnología diferente y puede utilizar distintos mecanismos de persistencia.

	\item \textbf{Maniobrabilidad en los despliegues}.
Ante cualquier cambio no es necesario desplegar la aplicación entera, solamente los microservicios implicados.

	\item \textbf{Escalabilidad y mantenibilidad}.
Los microservicios, y la separación funcional que otorgan, facilitan el escalado de las diferentes partes de la aplicación de manera independiente. Lo mismo sucede con el mantenimiento, pudiendo crear equipos especializados.

\end{itemize}

Se puede obtener más información acerca de los microservicios en el artículo de James Lewis y Martin Fowler titulado ''\emph{Microservices}'' \cite{LewisAndFowler}.

Por contraparte, este tipo de arquitecturas aumentan la complejidad del desarrollo en algunos aspectos concretos como pueden ser el versionado de los microservicios o la coordinación de las comunicaciones entre ellos. Es aquí donde entra este trabajo, pues está enfocado a paliar otra de sus desventajas: la exposición de múltiples puntos de entrada al \emph{backend}. Para ello, se quiere desarrollar un nuevo componente que actúe de intermediario. La figura \ref{figura:intermediario_o_no} muestra la comparativa de una arquitectura de microservicios basíca y otra que utiliza el componente mencionado.

\begin{figure}[ht]
	\centering
	\label{figura:intermediario_o_no}
	\includegraphics[width=0.45\textwidth]{images/arquitecturaMicroserviciosBasica}
	\includegraphics[width=0.45\textwidth]{images/arquitecturaMicroserviciosConIntermediario}
	\caption{Arquitectura de microservicios básica frente a una con intermediario. Imágenes de \url{https://www.adictosaltrabajo.com/2020/05/27/introduccion-al-api-gateway-pattern}.}
\end{figure}

\section{Motivación}

Este proyecto surge en el contexto de una práctica en empresa. El autor del mismo ha formado parte del equipo de \emph{I+D+i} de la compañía \emph{ADD Informática} durante el periodo de un año. \emph{ADD} es una empresa española que se dedica a desarrollar y comercializar un \emph{software} de gestión geriátrica llamado \emph{ResiPlus} \cite{ResiPlus}. Actualmente cuenta con casi cien trabajadores, es líder del mercado nacional y se encuentra en proceso de expansión internacional.

La temática del trabajo es el desarrollo de un intermediario entre la interfaz de usuario de una aplicación y sus microservicios. Esta fue elegida debido a la estrecha relación que guarda con el desempeño del autor en las prácticas mencionadas en el párrafo anterior. Por otro lado, el desarrollo a llevar a cabo le va a otorgar una visión más global de la aplicación sobre la que se trabaja así como aumentar el nivel de conocimiento acerca de la misma con la motivación de seguir contribuyendo al proyecto por mucho tiempo más. Por último, la tecnología a utilizar, \emph{.NET}, es de su interés y aspira así a crecer como desarrollador de ese \emph{framework}.

En cuanto a la motivación de la empresa, esta quiere dar la posibilidad de aprovechar, en cierta medida, el trabajo realizado y a realizar por el autor del documento dentro del marco de trabajo de la entidad, viéndose esta beneficiada también.

\section{Objetivos}

El objetivo principal es construir un \emph{proxy} inverso.
Este proyecto pretende resolver un problema real. Este consiste en otorgar a una aplicación con arquitectura de microservicios la capacidad de ocultarlos. Por otro lado, también se quiere permitir la ejecución simultánea de más de una instancia de los mismos. Cabe destacar que la aplicación en cuestión se encuentra todavía en fase de desarrollo y que el componente intermediario que se va a realizar como solución es un punto clave en su arquitectura. A raíz de lo comentado se obtienen los siguientes objetivos sobre la aplicación en construcción:

\begin{itemize}

	\item \textbf{Ocultar los microservicios} de forma que la interfaz de usuario no acceda directamente a estos por motivos de seguridad.
	
	\item Permitir la \textbf{instanciación múltiple} de los microservicios que, a su vez, tiene por objetivos:
		\begin{itemize}
		
			\item Conseguir que la aplicación sea \textbf{tolerante a fallos}, gracias a la posibilidad de tener un mismo microservicio desplegado en máquinas diferentes.
			
			\item \textbf{Aumentar la eficiencia}, al poder crear o parar instancias dinámicamente según el tráfico que reciba la aplicación.
			
		\end{itemize}

\end{itemize}

\section{Estructura del documento}

?????

%%%%%%%%%%%%%%%%%%%%%%%%%%%%%% ESTADO DEL ARTE %%%%%%%%%%%%%%%%%%%%%%%%%%%%%%

\chapter{Estado del arte}

En la actualidad existen en el mercado muchas aplicaciones y servicios que resuelven problemas parecidos al planteado. Estas herramientas se pueden dividir en dos tipos: los \emph{API Gateway} y los \emph{proxy} inversos. Ambos patrones comparten algunos casos de uso, por lo que sus diferencias causan confusión y no suelen quedar claras. Generalmente, se entiende que un \emph{API Gateway} es una especialización de un \emph{proxy} inverso, proporcionando así funcionalidades extra. Las más aceptadas e importantes son:

\begin{itemize}

	\item Interpretan los mensajes que reciben y pueden hacer transformaciones sobre ellos; los \emph{proxy} inversos solo los redirigen donde corresponda.
	\item Suelen ofrecer agregaciones de peticiones, esto es, aunar dos o más llamadas al \emph{backend} y exponer esta composición a través de un único \emph{endpoint}.
	\item Realizan tareas transversales a todos los microservicios como autenticación, autorización o monitorización.

\end{itemize}

Algunos de estos productos \emph{software} son de pago, otros gratuitos e incluso algunos de código abierto. Se comentan los más relevantes a continuación.

\section{\emph{Kong}}
Tal y como dice su página web, \emph{Kong} \cite{Kong} es el \emph{API Gateway} más conocido a nivel mundial. Cuenta con las siguientes características:

\begin{itemize}

	\item Está optimizado para arquitecturas de microservicios.
	\item Está concebido para ser altamente escalable.
	\item Ha sido creado sobre el \emph{proxy} inverso \emph{NGINX}, lo que hace que su eficiencia destaque.
	\item Es gratuito, aunque ofrece la versión \emph{Enterprise}, que facilita la gestión del API Gateway, y la \emph{SaaS}, llamada \emph{Kong Cloud}.
	\item Posee un gran número de \emph{plugins} desarrollados por la propia compañía y por su comunidad, que es bastante grande al ser una aplicación tan popular.
	\item Actualmente es utilizado por grandes empresas como \emph{Intel}, \emph{Ferrari} o \emph{Yahoo!}.

\end{itemize}

\section{\emph{Tyk}}
\emph{Tyk} \cite{Tyk} es un \emph{API Gateway} de código abierto.

\begin{itemize}

	\item Permite alojar sus servicios tanto en la nube como en una infraestructura propia. También ofrece un modelo híbrido.
	\item Dispone de extensiones para integrarlo con diferentes mecanismos de autenticación como \emph{OAuth} o \emph{LDPA}.
	\item Necesita una base de datos \emph{Redis}.
	\item Puede escalar horizontalmente y verticalmente.

\end{itemize}

\section{\emph{Ocelot}}
\emph{Ocelot} \cite{Ocelot} es una librería para \emph{.NET Core} que permite a una aplicación de ese \emph{framework} actuar como API Gateway. Posee las características siguientes:

\begin{itemize}

	\item Está pensada para arquitecturas orientadas a servicios o a microservicios.
	\item Al tratarse de una librería, se utiliza de forma sencilla añadiéndola como un paquete \emph{NuGet} más.
	\item Su configuración es muy básica, teniendo que especificarla en un fichero \emph{json}.
	\item No permite cambiar su configuración de manera dinámica.

\end{itemize}

\section{\emph{NGINX}}
Originariamente \emph{NGINX} \cite{NGINX} fue construido para ser un servidor web, como por ejemplo \emph{Apache}, pero más tarde ofreció la posibilidad de actuar como \emph{proxy} inverso, balanceador de carga o \emph{proxy} para protocolos de correo electrónico. Desde la vertiente que interesa a este trabajo:

\begin{itemize}

	\item Se trata de un \emph{proxy} inverso ligero y de alto rendimiento
	\item Ofrece una versión gratuita y otra de pago, \emph{NGINX Plus}, la cual ofrece funcionalidades extra.
	\item Se configura a través de un fichero el cual puede ser recargado durante su ejecución, es decir, es configurable dinámicamente.

\end{itemize}

\section{\emph{YARP}}
\emph{YARP} \cite{YARP} es otra librería para \emph{.NET} hecha por el propio \emph{Microsoft}. En este caso, facilita la creación de \emph{proxies} inversos. Sus aspectos más destacables son:

\begin{itemize}

	\item Se encuentra todavía en desarrollo y la única versión lanzada hasta la fecha es una \emph{preview}. Se prevé que vayan saliendo a la luz más versiones con más funcionalidades basadas en la experiencia de la propia empresa pero también en las opiniones de la gente.
	\item Fue diseñada para construir \emph{proxies} inversos dentro de la propia infraestructura de \emph{Microsoft} ya que en muchas áreas de la compañía necesitaban algo así.
	\item A raíz de la heterogeneidad de casos de uso que debe cubrir derivado del punto anterior, está pensada para ser muy personalizable y flexible.
	\item Permite cambiar su configuración de forma dinámica.

\end{itemize}

\section{Comparativa}

\section{Tecnología utilizada}

%%%%%%%%%%%%%%%%%%%%%%%%%%%%%% DESARROLLO DE LA SOLUCIÓN %%%%%%%%%%%%%%%%%%%%%%%%%%%%%%

\chapter{Desarrollo de la solución}

?????

%%%%%%%%%%%%%%%%%%%%%%%%%%%%%% ESPECIFICACIÓN DE REQUISITOS %%%%%%%%%%%%%%%%%%%%%%%%%%%%%%

\section{Especificación de requisitos}

?????

%%%%%%%%%%%%%%%%%%%%%%%%%%%%%% DISEÑO %%%%%%%%%%%%%%%%%%%%%%%%%%%%%%

\section{Diseño}

?????

%%%%%%%%%%%%%%%%%%%%%%%%%%%%%% PROGRAMACIÓN %%%%%%%%%%%%%%%%%%%%%%%%%%%%%%

\section{Programación}

?????

\subsection{Prototipo microservicio básico}

?????

\subsection{Prototipo microservicio autogenerado}

?????

\subsection{Comparativa de tiempos}

?????

%%%%%%%%%%%%%%%%%%%%%%%%%%%%%% PRUEBAS %%%%%%%%%%%%%%%%%%%%%%%%%%%%%%

\section{Pruebas}

?????

%%%%%%%%%%%%%%%%%%%%%%%%%%%%%% CONCLUSIONES %%%%%%%%%%%%%%%%%%%%%%%%%%%%%%

\chapter{Conclusiones}

?????

%%%%%%%%%%%%%%%%%%%%%%%%%%%%%% BIBLIOGRAFÍA %%%%%%%%%%%%%%%%%%%%%%%%%%%%%%

\begin{thebibliography}{10}

\bibitem{LewisAndFowler}
	Artíclo de J. Lewis y M. Fowler acerca de los microservicios.
	\newblock Disponible en:\\
	\url{https://martinfowler.com/articles/microservices.html}

\bibitem{ResiPlus}
	Página web de \emph{ADD Informática} y su producto \emph{ResiPlus}:\\
	\url{https://addinformatica.com}

\bibitem{Kong}
	Documentación oficial de \emph{Kong}:\\
	\url{https://konghq.com/kong}

\bibitem{Tyk}
	Documentación oficial de \emph{Tyk}:\\
	\url{https://tyk.io}

\bibitem{Ocelot}
	Documentación oficial de \emph{Ocelot}:\\
	\url{https://threemammals.com/ocelot}

\bibitem{NGINX}
	Documentación oficial de \emph{NGINX}:\\
	\url{https://www.nginx.com/}

\bibitem{YARP}
	Documentación oficial de \emph{YARP}:\\
	\url{https://microsoft.github.io/reverse-proxy}

\end{thebibliography}

\end{document}
